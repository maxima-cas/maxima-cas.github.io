%%
%% Este es un ejemplo simple (aunque util!) de uso
%% del paquete maxiplot (originalmente maxima, con modificaciones de J.M.Mira)
%% Jose Miguel M. Planas <nohaim@gmail.com>
%%
%% Para compilar:
%% > latex gauss.tex
%% > maxima -b gauss.mac
%% > gnuplot gauss.gnuplot
%% > latex gauss.tex
%%
%% Para mas informacion, visiten la pagina
%% http://webs.um.es/mira/tex/maxima_latex.php
%%


\documentclass[10pt,a4paper]{article}
\usepackage{graphicx}
\usepackage{amsmath}
\usepackage{anysize}
\usepackage{maxiplot}

\begin{document}
\pagestyle{empty}
\marginsize{2.7cm}{3cm}{2cm}{1.5cm}

\begin {gnuplot}
  set output "gauss.png";
  set term png crop enhanced;
  set style fill solid 0.25;
  set size 1,0.5;
  set parametric;
  f(x)=exp(-x**2/2)*0.3989422;
  set xrange [-3:3];
  set trange [-3:5];
  unset xtics;
  set xlabel "Z" offset 9,0;
  path(t) = ( t<=1 ? f(t) : 0 );
  fold(t) = ( t<=1 ? t : 2 - t);
  plot fold(t),path(t) notitle with filledcurves closed, t, f(t) notitle;
\end{gnuplot}

\begin{center}
{\Huge Tabla de la distribuci\'on Normal(0,1)}\\
\mxpIncludegraphics[scale=0.35]{gauss.png}
\vspace{-0.5em}
\[
    F(z) = P(Z \leq z) = \frac{1}{\sqrt{2\pi}}
    \int_{-\infty}^z e^{\frac{-x^2}{2}}dx
\]
\end{center}

\begin{maxima*}[tabla]
  numer:true,
  ratprint:false,
  g:exp(-x**2/2)/sqrt(2*%pi),
  grow(i):= block(
    fpprintprec:2,
    I:0.5 + integrate(g,x,0,i/10),
    L[0]:I,
    for j:1 while j<=9 do(
      I:I+(ev(g,x=i/10+(j-1)/100)+ev(g,x=i/10+j/100))*0.005,
      L[j]:I
      ),
    if i=0 then str:0.0 else str:i/10 + 0.001,
    for j:0 while j<=9 do(
      str:concat(str,"&\\small.",?round(100000*L[j]))
      ),
    print(str,"\\\\")
    ),
    for i:0 while i<40 do (grow(i))
\end{maxima*}
\begin{center}
\begin{tabular}{|c|cccccccccc|}
  \hline
  z&0.00&0.01&0.02&0.03&0.04&0.05&0.06&0.07&0.08&0.09\\
  \hline
  \tabla
  \hline
\end{tabular}
\end{center}
\end{document}
