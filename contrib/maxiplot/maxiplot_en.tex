\documentclass[11pt,a4paper]{article}
\usepackage[ansinew]{inputenc}
\usepackage{graphicx}
\usepackage{amsmath}
\usepackage[amsmath]{maxiplot}

\title{Maxiplot: Maxima and Gnuplot in \LaTeX.\\}
\date{September 21, 2013}

\def\Maxima{\emph{Maxima}}
\def\Gnuplot{\emph{Gnuplot}}

\begin{document}
\maketitle

\section{Introduction}
For those who do not know \Maxima, it is a symbolic calculation
program which can be used to compute derivatives and integrals, solve
equations, find limits, work with vectors and matrices and create
graphics, among many other things. It also adds the possibility to
write programs, thus expanding its capabilities. As if all this was
not enough, it is also released under the GNU General Public License
and it can be downloaded for free at
\texttt{http://maxima.sourceforge.net}, where there is also
documentation in several languages (including Spanish).

The purpose of this \LaTeX{} package is to provide ``programming''
capabilities importing the results, without the need of working with
various files and interfaces. Maxima code can be included within the
\LaTeX{} document. When the document is processed, a file with
extension \texttt{.mac} is generated, which can be directly processed
by Maxima, creating another file with extension \texttt{.mxp}; when
the \LaTeX{} document is processed again, that file will be
automatically inserted.

\Gnuplot{} commands can also be inserted, thanks to some additional
commands added by J. M. Mira. Thus, in addition to the auxiliary files
already mentioned, another file with extension \texttt{.gnp} will be
created, which after being processed by \Gnuplot{} can be added to the
document.

\section{Installation}

Just copy the file \texttt{maxiplot.sty} into a place where \LaTeX{}
can find it, or copy it into the same directory where you have your
document. For those who have used previous versions of
\texttt{maxiplot}, notice that in this version \emph{no other files are
  needed}.

\section{ \LaTeX{} package \texttt{maxiplot}}

\subsection{How is it used?}
Its usage is simple. Process your document as normally done; for
instance, write in the command line:

\texttt{latex mydocument.tex}

You will find out that a file \texttt{mydocument.mac} has been
created in your working directory. Process that file with Maxima:

\texttt{maxima -b mydocument.mac}

And if you have used \Gnuplot{} commands in your document:

\texttt{gnuplot mydocument.gnp}

Process your \LaTeX{} document again, \textit{et voil\`a!}.

If your distribution allows it, you can enable the command
\texttt{write18} to have \Maxima{} and \Gnuplot{} automatically run
when you process your \LaTeX{} document (you should previously add
your installation directory to the executable search path in you
operating system). 

\subsection{User interface}
\subsubsection{Maxima.}
This section and the following ones show some examples of the use of
package \texttt{maxiplot}. It would be convenient to have some basic
knowledge of Maxima to follow the examples.

This package has an option (for the time being) to allow compatibility
with the \texttt{pmatrix} environment of the \texttt{amsmath}
package. Therefore, if you are going to create matrices with that
environment, you should add the following lines to the document preamble
\begin{verbatim}
\usepackage{amsmath}
\usepackage[amsmath]{maxiplot}
\end{verbatim}

The most important environments are \texttt{maxima} and
\texttt{maximacmd}. The contents of those environments will be passed
to a file with extension \texttt{.mac} to be processed later on with
Maxima. Hence, there can be no \LaTeX-style comments within those
environments; namely, \% cannot be used to start a comment since that
symbol has a special meaning in Maxima. Instead, comments within those
environment should follow the C language syntax (/* \textit{comment}
*/). Commands will be inserted as arguments for a function; therefore,
they must be separated by commas.

\pagebreak
Let us start with a simple example:
\begin{verbatim}
\[   %Math mode begin
\begin{maxima}
  f: x/(x^3-3*x+2),     /* Integrating it */
  tex('integrate(f,x)), /* will show its integral... */
  print("="),
  tex(integrate(f,x)),  /* ...and the result */
  print("+K")
\end{maxima}
\]   %Math mode end
\end{verbatim}

In the place with this code is found, the result will be:

\[
\begin{maxima}
  f: x/(x^3-3*x+2),     
  tex('integrate(f,x)), 
  print("="),
  tex(integrate(f,x)),  
  print("+K")
\end{maxima}
\]
\vskip1em
There are some environments where a \texttt{maxima} block cannot be
included. In those cases the \texttt{maxima*}{} variant can be used,
which gives output immediately. That output can be then inserted with
the command \verb|\maximacurrent|, as in the following example:
\begin{verbatim}
\begin{maxima*}
  suml(L):=lsum(i,i,L), 
  printrow(L):=block(
    [str:""],
    for i:1 step 1 thru length(L)-1 do(
        str:concat(str,L[i],"&")),
    str:concat(str,L[length(L)],"\\\\"),
    print(str)),
  xi:[1,2,3,4,5,6],
  fi:[3,4,7,10,8,2],
  for i:1 while i<=length(xi) do (
    printrow([xi[i],fi[i],(fi*xi)[i],(fi*xi^2)[i]])
    ),
  print("\\hline"),
  printrow(["",N:suml(fi),fx:suml(fi*xi),fx2:suml(fi*xi^2)])
\end{maxima*}
                 
\begin{center}
  \begin{tabular}{|c|c|c|c|c|}
  $x_i$&$n_i$&$n_i\cdot x_i$&$n_i\cdot x_i^2$\\
  \hline
  \maximacurrent
  \end{tabular}
\end{center}
\end{verbatim}

\begin{maxima*}
  suml(L):=lsum(i,i,L), 
  printrow(L):=block(
    [str:""],
    for i:1 step 1 thru length(L)-1 do(
        str:concat(str,L[i],"&")),
    str:concat(str,L[length(L)],"\\\\"),
    print(str)),
  xi:[1,2,3,4,5,6],
  fi:[3,4,7,10,8,2],
  for i:1 while i<=length(xi) do (
    printrow([xi[i],fi[i],(fi*xi)[i],(fi*xi^2)[i]])
    ),
  print("\\hline"),
  printrow(["",N:suml(fi),fx:suml(fi*xi),fx2:suml(fi*xi^2)])
\end{maxima*}
                 
\begin{center}
  \begin{tabular}{|c|c|c|c|c|}
  $x_i$&$n_i$&$n_i\cdot x_i$&$n_i\cdot x_i^2$\\
  \hline
  \maximacurrent
  \end{tabular}
\end{center}

It is important to keep in mind that the \verb|\maximacurrent| command
will be replaced by the result of \emph{the last \texttt{maxima}
  block}, so it must be used before any other \texttt{maxima} blocks.

If you would like to use that result later on, or if you are going to
use it at several places in the document, you can add an optional
command with a variable name that will save that content. The previous
example might as well had been implemented in the following way:
\begin{verbatim}
\begin{maxima*}[table]
  suml(L):=lsum(i,i,L), 
  printrow(L):=block(
    [str:""],
    for i:1 step 1 thru length(L)-1 do(
        str:concat(str,L[i],"&")),
    str:concat(str,L[length(L)],"\\\\"),
    print(str)),
  xi:[1,2,3,4,5,6],
  fi:[3,4,7,10,8,2],
  for i:1 while i<=length(xi) do (
    printrow([xi[i],fi[i],(fi*xi)[i],(fi*xi^2)[i]])
    ),
  print("\\hline"),
  printrow(["",N:suml(fi),fx:suml(fi*xi),fx2:suml(fi*xi^2)])
\end{maxima*}
                 
\begin{center}
  \begin{tabular}{|c|c|c|c|c|}
  $x_i$&$n_i$&$n_i\cdot x_i$&$n_i\cdot x_i^2$\\
  \hline
  \table
  \end{tabular}
\end{center}
\end{verbatim}

Notice that when passing ``\texttt{table}'' as a parameter to
\texttt{maxima*} \emph{there is no need to use a backslash} (\verb|\|).

There is a line-mode version of the \texttt{maxima} environment,
with similar usage and options: the command \verb|\imaxima| (from
``inline maxima'').
\begin{verbatim}
  \[
  \overline{x}=\imaxima{tex(xx:fx/N)}\qquad
  \sigma^2=\imaxima{tex(sx2:fx2/N-xx^2)}\qquad
  \sigma=\imaxima{tex(sqrt(sx2))}
  \]       
\end{verbatim}

\[ 
\overline{x}=\imaxima{tex(xx:fx/N)}\qquad
\sigma^2=\imaxima{tex(sx2:fx2/N-xx^2)}\qquad
\sigma=\imaxima{tex(sqrt(sx2))}
\]

In cases when no output is expected, such as defining a function or
loading \Maxima{} packages, the \texttt{maximacmd} environment or
\verb|\imaximacmd| command should be used. These two do not have
\texttt{*} variant nor any options. Furthermore, the \Maxima{}
commands inside these must be separated by a semicolon (;) or, better
yet, a dollar sign (\$). As an example, let us look at some of the
features of the \Maxima/\Gnuplot{} interface. This example shows the
plots of the $\sin$ function and its tangent at $\frac{\pi}{3}$:

\begin{verbatim}
\begin{maximacmd}
  tangent(fx,a):=expand(ev(fx,x=a)
                  +subst(a,x,diff(fx,x))*(x-a))$
  plot2d([sin(x),tangent(sin(x),%pi/3)], [x,-3,3],
         [gnuplot_preamble,"set zeroaxis;"],
         [gnuplot_term, png],
         [gnuplot_out_file,"./\jobname2D.png"])$
\end{maximacmd}
\begin{center}
  \mxpIncludegraphics[scale=0.60]{\jobname2D.png}
\end{center}
\end{verbatim}   

\pagebreak
This code creates a \verb|png| format file \texttt{\jobname2D.png}:
\begin{maximacmd}
  tangent(fx,a):=expand(ev(fx,x=a)
                  +subst(a,x,diff(fx,x))*(x-a))$
  plot2d([sin(x),tangent(sin(x),%pi/3)], [x,-3,3],
         [gnuplot_preamble,"set zeroaxis;"],
         [gnuplot_term, png],
         [gnuplot_out_file,"./\jobname2D.png"])$
\end{maximacmd}
\begin{center}
  \mxpIncludegraphics[scale=0.60]{\jobname2D.png}
\end{center}

The environments introduced so far can contain \LaTeX{} commands which
will be replaced before passing them to the \texttt{mac}
file. Sometimes this feature might not be desired and could lead to
problems with certain strings. The environments \texttt{vmaxima} and
\texttt{vmaximacmd} solve this problem; their usage is similar to the
previous environments, but their content is passed literally. These
environments are based on the \texttt{verbatim} \LaTeX{} package.

\pagebreak

\subsubsection{Gnuplot}
While \Maxima{} can create graphics via \Gnuplot{}, sometimes it might
be preferable to work directly with this last program. In order to do
that, the environments \texttt{gnuplot} and its verbatim version
\texttt{vgnuplot} are used.

Here is a 3D example
\begin{verbatim}
\begin{gnuplot}
  set term png crop enhanced font "calibri, 10"
  set output "toros.png"
  set parametric
  set urange [0:2*pi]
  set vrange [-pi:pi]
  set isosamples 36,24
  set hidden3d
  set view 75,15,1,1
  unset key
  set ticslevel 0
  x1(u,v)=cos(u)+.5*cos(u)*cos(v)
  y1(u,v)=sin(u)+.5*sin(u)*cos(v)
  z1(u,v)=.5*sin(v)
  x2(u,v)=1+cos(u)+.5*cos(u)*cos(v)
  y2(u,v)=.5*sin(v)
  z2(u,v)=sin(u)+.5*sin(u)*cos(v)
  set multiplot
  splot x1(u,v), y1(u,v), z1(u,v) w pm3d, x2(u,v), y2(u,v), z2(u,v) w pm3d
  splot x1(u,v), y1(u,v), z1(u,v) lt 3,   x2(u,v), y2(u,v), z2(u,v) lt 5 
\end{gnuplot}
\begin{center}
  \mxpIncludegraphics[scale=0.75]{toros.png}
\end{center}
\end{verbatim}


\begin{gnuplot}
  set term png crop enhanced font "calibri, 10"
  set output "toros.png"
  set parametric
  set urange [0:2*pi]
  set vrange [-pi:pi]
  set isosamples 36,24
  set hidden3d
  set view 75,15,1,1
  unset key
  set ticslevel 0
  x1(u,v)=cos(u)+.5*cos(u)*cos(v)
  y1(u,v)=sin(u)+.5*sin(u)*cos(v)
  z1(u,v)=.5*sin(v)
  x2(u,v)=1+cos(u)+.5*cos(u)*cos(v)
  y2(u,v)=.5*sin(v)
  z2(u,v)=sin(u)+.5*sin(u)*cos(v)
  set multiplot
  splot x1(u,v), y1(u,v), z1(u,v) w pm3d, x2(u,v), y2(u,v), z2(u,v) w pm3d
  splot x1(u,v), y1(u,v), z1(u,v) lt 3,   x2(u,v), y2(u,v), z2(u,v) lt 5 
\end{gnuplot}
\begin{center}
  \mxpIncludegraphics[scale=0.75]{toros.png}
\end{center}

Let us examine the \verb|\mxpIncludegraphics| command: its usage is
the same as \verb|includegraphics| from package \verb|graphicx|; in
fact, it just makes sure that the graphic file exists before invoking
that macro.

\subsection{Problems}
This is an experimental version; many of the capabilities of Maxima
have not been tested and it has not been tried with the most important
\LaTeX{} packages. Thus, it will surely need some tweaking.

However, I think that most of the problems will appear when showing
certain outputs. For instance, if the result of a computation is too
long, it will not be easy to break it into several lines (except if
one works in Maxima and then copies the result to the document, of
course).

Other possible problems can be addressed from within the \LaTeX{}
document. By default, Maxima orders expressions by inverse
alphabetical order; hence, if we type:\\
\verb|  $$\imaxima{tex(x+y+z+t=0)}$$|\\
we get:
$$\imaxima{tex(x+y+z+t=0)}$$

That can be avoided by using Maxima functions \texttt{ordergreat} and
\texttt{unorder}:
\begin{verbatim}
\imaximacmd{ordergreat(x,y,z,t)$}
$$\imaxima{tex(x+y+z+t=0)}$$
\imaximacmd{unorder()$}
\end{verbatim}
%\imaximacmd{ordergreat(x,y,z,t)$}
%$$\imaxima{tex(x+y+z+t=0)}$$
%\imaximacmd{unorder()$}

Furthermore, if we would like to align several equations, we will need
to dive a little deeper:
\begin{verbatim}
\begin{maximacmd}
  ordergreat(x,y,z)$
  :lisp(defprop mequal (&=) texsym)
\end{maximacmd}

\begin{maxima*}
  eq1:a-2*b=x+y,
  eq2:b=2*x-3*y+2*z,
  tex(eq1),
  print("\\\\"),
  tex(eq2)
\end{maxima*}

\begin{maximacmd}
  unorder()$
  :lisp(defprop mequal (=) texsym)    
\end{maximacmd}
\end{verbatim}

\begin{maximacmd}
  ordergreat(x,y,z)$
  :lisp(defprop mequal (&=) texsym)
\end{maximacmd}

\begin{maxima*}
  eq1:a-2*b=x+y,
  eq2:b=2*x-3*y+2*z,
  tex(eq1),
  print("\\\\"),
  tex(eq2)
\end{maxima*}

\begin{maximacmd}
  unorder()$
  :lisp(defprop mequal (=) texsym)    
\end{maximacmd}

\begin{align}
  \maximacurrent
\end{align}

\section{A few last words}
As I mentioned before, this is an experimental package that will
probably need some amendments and additions, so any ideas or comments
will be welcome.
\\

\raggedleft{Jos\'e Miguel M. Planas}\\
\raggedleft{$<$nohaim@gmail.com$>$}\\[10pt]
\raggedleft{(English translation by Jaime Villate)}
                  
\end{document}