\documentclass[11pt,a4paper]{article}
\usepackage[spanish]{babel}
\usepackage[ansinew]{inputenc}
\usepackage{graphicx}
\usepackage{amsmath}
\usepackage[amsmath]{maxiplot}

\title{Maxiplot: Maxima y Gnuplot en \LaTeX.\\}
\date{21 de septiembre de 2013}

\def\Maxima{\emph{Maxima}}
\def\Gnuplot{\emph{Gnuplot}}

\begin{document}
\maketitle

\section{Introducci\'on.}
Para los que aun no conocen \Maxima, se trata de un programa de c\'alculo simb\'olico que 
permite, entre otras cosas, calcular derivadas, integrales, resolver ecuaciones y
l\'imites, manejo de vectores y matrices, generar gr\'aficos... Adem\'as es posible 
crear programas, pudiendo as\'i ampliar sus capacidades. Por si esto fuese poco, 
est\'a bajo licencia GNU, y se puede descargar gratuitamente desde 
\texttt{http://maxima.sourceforge.net}, donde tambi\'en se puede encontrar toda la 
documentaci\'on en varios idiomas (incluyendo el espa\~nol).

El prop\'osito de este paquete \LaTeX{} es, precisamente, permitir `programar' e importar
los resultados sin necesidad de trabajar con varios ficheros y entornos. Dentro 
del documento \LaTeX{} se podr\'a incluir c\'odigo en Maxima; al compilar el documento
se genera un fichero (con extensi\'on \texttt{.mac}) directamente procesable por Maxima,
que a su vez genera otro fichero (con extensi\'on \texttt{.mxp}) que, al recompilar el 
documento \LaTeX{}, ser\'a autom\'aticamente insertado.

De igual forma se pueden insertar comandos \Gnuplot{} gracias a los comandos a\~nadidos por J. M. Mira. As\'i, adem\'as de los ficheros anteriores, se generar\'a un fichero con la extensi\'on \texttt{.gnp} que, tras ser procesados por \Gnuplot{}
podr\'an agregarse a su documento.

\section{Instalaci\'on.}
Simplemente coloque el archivo \texttt{maxiplot.sty} en una ruta conocida para \LaTeX{} o en el mismo
directorio de su documento. Para los que hayan usado las primeras versiones del paquete \texttt{maxima}, decirles
que \emph{ya no es necesario ning\'un otro fichero}.

\section{El paquete \texttt{maxiplot} para \LaTeX{}.}

\subsection{?`C\'omo se usa?}
Sencillo. Compile normalmente su documento, por ejemplo, desde la linea de comandos:

\texttt{latex midocumento.tex}

Encontrar\'a entonces en su directorio de trabajo que se ha generado un archivo
\texttt{midocumento.mac}. Procese este documento con Maxima:

\texttt{maxima -b midocumento.mac}

Y, si contiene comandos \Gnuplot:

\texttt{gnuplot midocumento.gnp}

Vuelva ahora a compilar su documento \LaTeX{} \textit{et voil\`a!}.

Si su distribuci\'on lo permite puede activar el comando \texttt{write18} que permitir\'a
que \Maxima{} y \Gnuplot{} se ejecuten autom\'aticamente al compilar su documento \LaTeX{} (previamente
deber\'a a\~nadir el directorio de su instalaci\'on a la ruta de b\'usqueda
de su sistema operativo). 

\subsection{La interfaz de usuario.}
\subsubsection{Maxima.}
En esta secci\'on y las siguientes se presentan algunos ejemplos de uso del paquete \texttt{maxiplot}.
Es conveniente que el usuario tenga alg\'un conocimiento b\'asico sobre Maxima.

Este paquete admite (por el momento) una opci\'on, que permite la compatibilidad
con el entorno \texttt{pmatrix} del paquete \texttt{amsmath}. As\'i, si va a usar
matrices en dicho entorno deber\'a especificar
\begin{verbatim}
\usepackage{amsmath}
\usepackage[amsmath]{maxiplot}
\end{verbatim}

Los entornos m\'as importantes son \texttt{maxima} y \texttt{maximacmd}.
El contenido de estos entornos ser\'a copiado a un archivo con extensi\'on \texttt{.mac}
para ser procesado despu\'es por Maxima. As\'i, no podr\'a contener comentarios 
al estilo \LaTeX{} (esto es, comenzando por \%), puesto que este s\'imbolo es utilizado 
por Maxima, sino que se insertan como en C (/* \textit{comentario} */). Los comandos
ser\'an insertados como argumentos de una funci\'on, as\'i que deber\'an ir separados por comas.
\pagebreak

Comencemos por un ejemplo sencillo:
\begin{verbatim}
\[   %Comienzo modo matem\'aticas
\begin{maxima}
  f: x/(x^3-3*x+2),     /* Integrando */
  tex('integrate(f,x)), /* Presenta la integral... */
  print("="),
  tex(integrate(f,x)),  /* ...y el resultado */
  print("+K")
\end{maxima}
\]   %F\'in modo matem\'aticas
\end{verbatim}

En el lugar donde coloquemos este c\'odigo obtendremos:

\[
\begin{maxima}
  f: x/(x^3-3*x+2),     
  tex('integrate(f,x)), 
  print("="),
  tex(integrate(f,x)),  
  print("+K")
\end{maxima}
\]
\vskip1em
Hay entornos dentro de los cuales no podr\'a incluir un bloque \texttt{maxima}. Para estos
casos puede usar la version \texttt{maxima*}{}, el cual no produce salida inmediata. 
Posteriormente podr\'a insertar dicha salida con el comando \verb|\maximacurrent|:
\begin{verbatim}
\begin{maxima*}
  suml(L):=lsum(i,i,L), 
  printrow(L):=block(
    [str:""],
    for i:1 step 1 thru length(L)-1 do(
        str:concat(str,L[i],"&")),
    str:concat(str,L[length(L)],"\\\\"),
    print(str)),
  xi:[1,2,3,4,5,6],
  fi:[3,4,7,10,8,2],
  for i:1 while i<=length(xi) do (
    printrow([xi[i],fi[i],(fi*xi)[i],(fi*xi^2)[i]])
    ),
  print("\\hline"),
  printrow(["",N:suml(fi),fx:suml(fi*xi),fx2:suml(fi*xi^2)])
\end{maxima*}
                 
\begin{center}
  \begin{tabular}{|c|c|c|c|c|}
  $x_i$&$n_i$&$n_i\cdot x_i$&$n_i\cdot x_i^2$\\
  \hline
  \maximacurrent
  \end{tabular}
\end{center}
\end{verbatim}

\begin{maxima*}
  suml(L):=lsum(i,i,L), 
  printrow(L):=block(
    [str:""],
    for i:1 step 1 thru length(L)-1 do(
        str:concat(str,L[i],"&")),
    str:concat(str,L[length(L)],"\\\\"),
    print(str)),
  xi:[1,2,3,4,5,6],
  fi:[3,4,7,10,8,2],
  for i:1 while i<=length(xi) do (
    printrow([xi[i],fi[i],(fi*xi)[i],(fi*xi^2)[i]])
    ),
  print("\\hline"),
  printrow(["",N:suml(fi),fx:suml(fi*xi),fx2:suml(fi*xi^2)])
\end{maxima*}
                 
\begin{center}
  \begin{tabular}{|c|c|c|c|c|}
  $x_i$&$n_i$&$n_i\cdot x_i$&$n_i\cdot x_i^2$\\
  \hline
  \maximacurrent
  \end{tabular}
\end{center}

Es importante tener en cuenta que el comando \verb|\maximacurrent| ser\'a sustituido
por el resultado \textit{del \'ultimo bloque \texttt{maxima}}, as\'i que deber\'a ser
usado \textit{antes} del siguiente bloque. Si quiere usar dicha salida posteriormente,
o si la va a insertar varias veces en su documento, puede usar como par\'ametro opcional el nombre
de un comando que almacenar\'a su contenido. En tal caso podr\'iamos implementar el 
ejemplo anterior como:
\begin{verbatim}
\begin{maxima*}[tabla]
  suml(L):=lsum(i,i,L), 
  printrow(L):=block(
    [str:""],
    for i:1 step 1 thru length(L)-1 do(
        str:concat(str,L[i],"&")),
    str:concat(str,L[length(L)],"\\\\"),
    print(str)),
  xi:[1,2,3,4,5,6],
  fi:[3,4,7,10,8,2],
  for i:1 while i<=length(xi) do (
    printrow([xi[i],fi[i],(fi*xi)[i],(fi*xi^2)[i]])
    ),
  print("\\hline"),
  printrow(["",N:suml(fi),fx:suml(fi*xi),fx2:suml(fi*xi^2)])
\end{maxima*}
                 
\begin{center}
  \begin{tabular}{|c|c|c|c|c|}
  $x_i$&$n_i$&$n_i\cdot x_i$&$n_i\cdot x_i^2$\\
  \hline
  \tabla 
  \end{tabular}
\end{center}
\end{verbatim}

Observemos que al pasar `\texttt{tabla}' como par\'ametro \textit{no
debe de usarse la barra invertida} (\verb|\|).

Existe una versi\'on `en linea' del entorno \texttt{maxima}, cuyo uso y opciones 
es similar: el comando \verb|\imaxima| (de `inline maxima').
\begin{verbatim}
  \[
  \overline{x}=\imaxima{tex(xx:fx/N)}\qquad
  \sigma^2=\imaxima{tex(sx2:fx2/N-xx^2)}\qquad
  \sigma=\imaxima{tex(sqrt(sx2))}
  \]       
\end{verbatim}

\[ 
\overline{x}=\imaxima{tex(xx:fx/N)}\qquad
\sigma^2=\imaxima{tex(sx2:fx2/N-xx^2)}\qquad
\sigma=\imaxima{tex(sqrt(sx2))}
\]

Para los casos en que no se espere salida alguna, sino que s\'olo se pretenda
definir funciones, o cargar paquetes de \Maxima, tenemos el entorno \texttt{maximacmd}
y su correspondiente comando \verb|\imaximacmd|. \'Estos no admiten la versi\'on
\texttt{*} ni par\'ametro alguno, ya que no son necesarios. Adem\'as los comandos 
\Maxima{} usados aqu\'i deber\'an ir separados por `punto y coma' (;) o, mejor por el s\'imbolo
`d\'olar' (\$). Como ejemplo, veamos algunas capacidades de la interfaz \Maxima/\Gnuplot.
En este ejemplo vemos las gr\'aficas de la funci\'on $seno$ y su tangente en $\frac{\pi}{3}$:

\begin{verbatim}
\begin{maximacmd}
  tangente(fx,a):=expand(ev(fx,x=a)
                  +subst(a,x,diff(fx,x))*(x-a))$
  plot2d([sin(x),tangente(sin(x),%pi/3)], [x,-3,3],
         [gnuplot_preamble,"set zeroaxis;"],
         [gnuplot_term, png],
         [gnuplot_out_file,"./\jobname2D.png"])$
\end{maximacmd}
\begin{center}
  \mxpIncludegraphics[scale=0.60]{\jobname2D.png}
\end{center}
\end{verbatim}   

\pagebreak
Este c\'odigo genera el archivo en formato \verb|png| \texttt{\jobname2D.png}:
\begin{maximacmd}
  tangente(fx,a):=expand(ev(fx,x=a)
                  +subst(a,x,diff(fx,x))*(x-a))$
  plot2d([sin(x),tangente(sin(x),%pi/3)], [x,-3,3],
         [gnuplot_preamble,"set zeroaxis;"],
         [gnuplot_term, png],
         [gnuplot_out_file,"./\jobname2D.png"])$
\end{maximacmd}
\begin{center}
  \mxpIncludegraphics[scale=0.60]{\jobname2D.png}
\end{center}

Los entornos presentados hasta ahora pueden contener comandos \LaTeX{} que ser\'an
sustituidos antes de pasar al archivo \texttt{mac}. Ahora bien, esta capacidad
puede, a veces, no ser deseable o puede dar problemas con algunas secuencias
de caracteres. Para evitar esto est\'an los entornos \texttt{vmaxima} y \texttt{vmaximacmd},
cuyo uso es similar a los anteriores, pero el volcado es literal. Estos entornos
est\'an basados en el paquete \texttt{verbatim} de \LaTeX{}. 

\pagebreak

\subsubsection{Gnuplot.}
Si bien \Maxima{} permite la inclusi\'on de gr\'aficos a trav\'es de \Gnuplot{}, a veces puede ser preferible trabajar directamente con \'este \'ultimo. Para esto usaremos los entornos \texttt{gnuplot} y su versi\'on `verbatim' \texttt{vgnuplot}.

Un ejemplo 3D
\begin{verbatim}
\begin{gnuplot}
  set term png crop enhanced font "calibri, 10"
  set output "toros.png"
  set parametric
  set urange [0:2*pi]
  set vrange [-pi:pi]
  set isosamples 36,24
  set hidden3d
  set view 75,15,1,1
  unset key
  set ticslevel 0
  x1(u,v)=cos(u)+.5*cos(u)*cos(v)
  y1(u,v)=sin(u)+.5*sin(u)*cos(v)
  z1(u,v)=.5*sin(v)
  x2(u,v)=1+cos(u)+.5*cos(u)*cos(v)
  y2(u,v)=.5*sin(v)
  z2(u,v)=sin(u)+.5*sin(u)*cos(v)
  set multiplot
  splot x1(u,v), y1(u,v), z1(u,v) w pm3d, x2(u,v), y2(u,v), z2(u,v) w pm3d
  splot x1(u,v), y1(u,v), z1(u,v) lt 3,   x2(u,v), y2(u,v), z2(u,v) lt 5 
\end{gnuplot}
\begin{center}
  \mxpIncludegraphics[scale=0.75]{toros.png}
\end{center}
\end{verbatim}


\begin{gnuplot}
  set term png crop enhanced font "calibri, 10"
  set output "toros.png"
  set parametric
  set urange [0:2*pi]
  set vrange [-pi:pi]
  set isosamples 36,24
  set hidden3d
  set view 75,15,1,1
  unset key
  set ticslevel 0
  x1(u,v)=cos(u)+.5*cos(u)*cos(v)
  y1(u,v)=sin(u)+.5*sin(u)*cos(v)
  z1(u,v)=.5*sin(v)
  x2(u,v)=1+cos(u)+.5*cos(u)*cos(v)
  y2(u,v)=.5*sin(v)
  z2(u,v)=sin(u)+.5*sin(u)*cos(v)
  set multiplot
  splot x1(u,v), y1(u,v), z1(u,v) w pm3d, x2(u,v), y2(u,v), z2(u,v) w pm3d
  splot x1(u,v), y1(u,v), z1(u,v) lt 3,   x2(u,v), y2(u,v), z2(u,v) lt 5 
\end{gnuplot}
\begin{center}
  \mxpIncludegraphics[scale=0.75]{toros.png}
\end{center}

Observemos el comando \verb|\mxpIncludegraphics|: el uso es el mismo que \verb|includegraphics| del
paquete \verb|graphicx|, de hecho lo que hace es asegurarse de que existe el archivo gr\'afico e invocar dicha macro.

\subsection{Problemas.}
Esta es a\'un una versi\'on en pruebas, y todav\'ia no han sido probadas muchas de las
capacidades de Maxima, ni se ha comprobado con los paquetes m\'as importantes de
\LaTeX{}, as\'i que seguramente necesite unos cuantos retoques.

Aunque a mi parecer los mayores problemas aparezcan a la hora de presentar algunas
salidas. Por ejemplo, si el resultado de una operaci\'on es muy largo no ser\'a f\'acil
dividir en varias lineas (salvo, claro est\'a, trabajando en Maxima y luego
copiando en el documento).

Otros `problemas' s\'i que se pueden atajar desde \LaTeX{}. Por defecto Maxima
ordena las expresiones en orden alfab\'etico inverso, as\'i si escribimos:
\verb|  $$\imaxima{tex(x+y+z+t=0)}$$|\\
obtendremos: 
$$\imaxima{tex(x+y+z+t=0)}$$

Para evitarlo podemos usar las funciones Maxima \texttt{ordergreat} y \texttt{unorder}:
\begin{verbatim}
\imaximacmd{ordergreat(x,y,z,t)$}
$$\imaxima{tex(x+y+z+t=0)}$$
\imaximacmd{unorder()$}
\end{verbatim}
%\imaximacmd{ordergreat(x,y,z,t)$}
%$$\imaxima{tex(x+y+z+t=0)}$$
%\imaximacmd{unorder()$}

Si adem\'as queremos alinear varias ecuaciones, tendremos que meternos un poco
m\'as en profundidad:
\begin{verbatim}
\begin{maximacmd}
  ordergreat(x,y,z)$
  :lisp(defprop mequal (&=) texsym)
\end{maximacmd}

\begin{maxima*}
  eq1:a-2*b=x+y,
  eq2:b=2*x-3*y+2*z,
  tex(eq1),
  print("\\\\"),
  tex(eq2)
\end{maxima*}

\begin{maximacmd}
  unorder()$
  :lisp(defprop mequal (=) texsym)    
\end{maximacmd}
\end{verbatim}

\begin{maximacmd}
  ordergreat(x,y,z)$
  :lisp(defprop mequal (&=) texsym)
\end{maximacmd}

\begin{maxima*}
  eq1:a-2*b=x+y,
  eq2:b=2*x-3*y+2*z,
  tex(eq1),
  print("\\\\"),
  tex(eq2)
\end{maxima*}

\begin{maximacmd}
  unorder()$
  :lisp(defprop mequal (=) texsym)    
\end{maximacmd}

\begin{align}
  \maximacurrent
\end{align}

\section{Para terminar:}
Como he dicho antes, este es un paquete en pruebas y muy posiblemente necesite
correcciones y adiciones, as\'i que cualquier idea o comentario ser\'a bienvenido.
\\

\raggedleft{Jos\'e Miguel M. Planas}\\
\raggedleft{$<$nohaim@gmail.com$>$}
                  
\end{document}
